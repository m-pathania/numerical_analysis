\documentclass{article}

\usepackage[T1]{fontenc}
\usepackage[margin=1in]{geometry}
\usepackage{amsmath}
\usepackage{graphicx}
\usepackage{algorithm}
\usepackage{algpseudocode}
\usepackage{caption}


\title{
    \textbf{Assignment 06\\}
    Ordinary Differential Equations\\
    Euler Method\\}

\author{Mayank Pathania\\204103314}

\begin{document}
    \maketitle
    \section{Given Problem}
        \[y^{'}\;=\;-100y\;+100t\;+101\]
        \[y(0)\;=1\]
    \section{Solution}
        \begin{center}
                \begin{tabular}{|c|c|c|c|c|}
                    \hline
                    t & Analytical & y(0) = 1 & y(0) = 0.99 & y(0) = 1.01\\
                    \hline
                    0 & 1 & 1 & 0.99 & 1.01\\
                    \hline
                    0.1	& 1.1 & 1.1 & 1.19 & 1.01\\
                    \hline
                    0.2	& 1.2 & 1.2 & 0.39 & 2.01\\
                    \hline
                    0.3	& 1.3 & 1.3 & 8.59 & -5.99\\
                    \hline
                    0.4	& 1.4 & 1.4 & -64.21 & 67.01\\
                    \hline
                    0.5	& 1.5 & 1.5 & 591.99 & -588.99\\
                    \hline
                    0.6	& 1.6 & 1.6 & -5312.81 & 5316.01\\
                    \hline
                    0.7	& 1.7 & 1.7 & 47831.4 & -47828\\
                    \hline
                    0.8	& 1.8 & 1.8 & -430465 & 430469\\
                    \hline
                    0.9	& 1.9 & 1.9 & 3.87421e+06 & -3.8742e+06\\
                    \hline
                    1 & 2 & 2 & -3.48678e+07 & 3.48678e+07\\
                    \hline
                \end{tabular}
        \end{center}
    
    \section{Conclusion \textbackslash Observation}
    \begin{itemize}
        \item The analytical solution of the problem is of the form \[y\;=\;t\;+\;1\;+\;\frac{C_1}{e^{100t}}\] where the $C_1$ is constant and depend on the initial condition. for  y = 1 at t = 0 $\implies\;C_1$ = 0. So the exponential term vanishes.
        \item For other initial conditions the exponential term remains and h = 0.1 is large step size to approximate y with exponential term which results in high error.
    \end{itemize}
\end{document}